\title{
  素早い茶色の狐が怠けた犬の上を飛ぶ
  %% RaspberryCom*PoTE の設計実装
  %% %% と使われなくなった情報コンセントの活用事例
}

\etitle{
  The quick brown fox jumps over the lazy dog
%% Design and Implemntasion of RaspberryCom*PoTE
}

\affiliate{Kanazawa}{金沢大学 総合メディア基盤センター\\
Kakuma-machi, Kanazawa, Ishikawa 920-1192, Japan}
\affiliate{USP}{ユニバーサル・シェル・プログラミング研究所}

\author{大野 浩之}{Hiroyuki Ohno}{Kanazawa}[hohno@staff.kanazawa-u.ac.jp]


\begin{comment}
概要(90%)
\end{comment}

\begin{abstract}
概要だよ 概要だよ 概要だよ 概要だよ 概要だよ 概要だよ 概要だよ 概要だよ 概要だよ 概要だよ 
\end{abstract}

\begin{jkeyword}
IoT, ものづくり, ものグラミング, Raspberry Pi, Arduino, POSIX, POSIX中心主義
\end{jkeyword}


\begin{eabstract}
abstract, abstract, abstract, abstract, abstract, abstract, abstract, abstract, abstract, abstract,
\end{eabstract}

\begin{ekeyword}
IoT, Mono-gramming, Monogramming, Raspberry Pi, Arduino, POSIX, POSIX centrics approach
\end{ekeyword}

\maketitle

\begin{comment}
1. はじめに(00%)
\end{comment}
\section{はじめに}
\label{sec:01introduction}


\begin{comment}
2. 背景(00%)
\end{comment}
\section{背景}
\label{sec:02background}


\begin{comment}
3.関連研究(00%)
\end{comment}
\section{関連研究}
\label{sec:03relatedworks}


\begin{comment}
4. 設計と実装(00%)
\end{comment}
\section{設計と実装}
\label{sec:04design_and_implementation}


\begin{comment}
5. 評価
\end{comment}
\section{評価}
\label{sec:05evaluation}


\begin{comment}
6. 今後の展開(00%)
\end{comment}
\section{今後の展開}
\label{sec:06nextstep}


\begin{comment}
7. おわりに(00%)
\end{comment}
\section{おわりに}
\label{sec:07conclusion}

\begin{comment}
謝辞(100%)
\end{comment}

\begin{acknowledgment}
  本研究を遂行するにあたり,カナダ国ノバスコシア州立ダルハウジー大学コンピュータサイエンス学部の Prof. Sampalli および同教授の研究室の学生からは,「ものグラミング」の有効性について検討する際に多くの示唆を得た.
  また,ユニバーサル・シェル・プログラミング研究所の當仲寛哲所長をはじめとする同社の研究部門の方々との議論も有益であった.ここに記して感謝したい.
\end{acknowledgment}

\newpage

\begin{thebibliography}{2}

\bibitem{hohno:monogramming1}
  大野,森,北口,中村,松浦,石山,當仲,ものづくりのための「ものグラミング」と実践的教育環境の構築,DICOMO2016,1335-1340, 2016-07.

\bibitem{misc:POSIXdocs1}
  What is POSIX?,The Open Group (オンライン),
  \urlj{https://collaboration.opengroup.org/external/pasc.org/plato/}
  \refdatej{2019-02-04}

\bibitem{matsuura:POSIXcentrics}
  松浦智之,大野浩之,當仲寛哲,ソフトウェアの高い互換性と長い持続性を目指すPOSIX中心主義プログラミング,デジタルプラクティス 8(4),352-360,2017-10-15.

\bibitem{misc:WSL_arch_overview}
  Seth Juarez,Windows Subsystem for Linux: Architectural Overview (オンライン),
  \urlj{https://channel9.msdn.com/Blogs/Seth-Juarez/Windows-Subsystem-for-Linux-Architectural-Overview}
  \refdatej{2019-05-22}

\bibitem{misc:firmata}
  Firmata Library
  \urlj{https://www.arduino.cc/en/Reference/Firmata}
  \refdatej{2019-05-22}

\bibitem{misc:kotoriotoko}
  秘密結社シェルショッカー日本支部,恐怖!小鳥男 (オンライン),
  \urlj{https://github.com/ShellShoccar-jpn/kotoriotoko}
  \refdatej{2019-02-04}


\end{thebibliography}
